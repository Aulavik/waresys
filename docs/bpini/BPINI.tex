\documentclass{bakalarka}
\usepackage[utf8]{inputenc} 
\usepackage[czech]{babel}
\usepackage{ae}
\usepackage{fancyhdr}
\usepackage[pdftex]{graphicx}
\author{Jan Kohlíček}
\title{Skladové hospodářství pomocí RFID a RaspberryPi}
\titlet{}
\titlett{}
\university{Západočeská univerzita v Plzni}
\faculty{Fakulta aplikovaných věd}
\department{Katedra informatiky a výpočetní techniky}
\subject{Bakalářská práce}
\town{Plzeň}
\begin{document}



\pagestyle{fancy}
\renewcommand{\chaptermark}[1]{\markboth{\textit{#1}}{}}
\renewcommand{\sectionmark}[1]{\markright{\textit{#1}}{}}
\cfoot{\thepage}
\lhead{\leftmark}
\rhead{\rightmark}
\maketitle
\chapter*{Prohlášení}
\thispagestyle{empty}
Prohlašuji, že jsem bakalářskou práci vypracoval samostatně a výhradně s~použitím citovaných pramenů.
\vskip 1.5em
V Plzni dne \today
\vskip 0.7em
\hskip 9cm Jan Kohlíček
\chapter*{Abstract}
\thispagestyle{empty}
Text of abstract.
\tableofcontents
\pagestyle{fancy}
\renewcommand{\chaptermark}[1]{\markboth{\textit{#1}}{}}
\renewcommand{\sectionmark}[1]{\markright{\textit{#1}}{}}
\cfoot{\thepage}
\lhead{\leftmark}
\rhead{\rightmark}
\parskip 1em
\chapter{Úvod}
	

%\chapter{IoT}

\chapter{Čtečka RFID}

	\section{Součástky}

		\subsection{Raspberry Pi}

			\subsubsection{Operační systém}
				Raspbian,Windows 10 IoT Core

			\subsubsection{GPIO}


		\subsection{RFID}
			popis jak to funguje
	
	

	\section{Sestavení}

	\section{Načtení ID}



\chapter{Server}

\chapter{Cloud}

\chapter{Klient}



\chapter{Závěr}


\chapter*{Literatura}

\chapter*{Uživatelský manuál}


\appendix
\bibliographystyle{csplainnat}
\bibliography{bakalarka}
\end{document}
